\documentclass{article} % Change to the appropriate document class, i.e. report, book, article...
\usepackage{style}

\title{CSE168 assignment 2}
\author{Alisha Lawson, Hallgeir Lien}
\date{} %Uncomment to remove date from title

\begin{document}

\maketitle

\section{Bounding-box intersection}
We do intersections with axis-aligned boxes. We check the planes that bounds the box in each dimension. So we find $t_{min_x}$, $t_{max_x}$ and similar for y and z-dimensions. We then check the maximum of the minimum hit times (that is, $\max_{i\in x,y,z}\{t_{min_i}\}$, and the minimum of the maximum hit times. If the first is larger than the second, we don't have any intersections.

\section{Building the BVH}
We implemented a fairly efficient BVH building algorithm. When we call root.build(), we first find the corner points for the box by checking all the vertices in the object list. We then try to find the best plane to split the box with. We go through each of the three dimensions. For each dimension, we start by trying to put the splitting plane in the middle of the root node and then put the objects in one of two lists (call them "left" and "right"). We then initially go through these lists to find the corner points of these. 

Now, we start a binary search to improve our plane. We check the cost using the cost function

$$
C=N_1\cdot SA_1+N_2\cdot SA_2
$$

where $SA_i$ is the surface area of the child boxes, and $N_i$ is the number of triangles in them. Note that we actually use the real surface area of the boxes in order to get an accurate estimate of the cost. We observe that one term is always increasing and the other is always decreasing. So if there is a better splitting plane, it's in the part with the largest cost, so we try setting our splitting plane in the middle there. Next we check the objects' positions and move the objects that are on the wrong side of the plane to the other child node.

An observation that increased the efficiency of our algorithm is that all the triangles in the lower cost box doesn't need to be checked again ever, because we know that if there's a better splitting plane, it's in the other child box. So, we record this index. In the future, NO objects that come before that index in the child node's object list will be checked again when we find a new splitting plane. So in best case, we cut the number of objects we check each iteration in half. 

Also, instead of iterating through all the objects every search step to get the corner points, we check the position of the vertices of the triangles we move between the child nodes to see if we need to update the corner points we found initially.

Bottom line is, this search algorithm should be about as precise as having $n=2^d$ splitting planes and checking each individually, but with far lower time complexity.

\section{BVH intersection}
The bounding box intersection was explained above. Other than that, the BVH intersection is a pretty straight forward recursive algorithm. First see if we intersect the root node. If yes, intersect with the children of that node (that might be triangles or child nodes). If it's child nodes, check intersection with those, and recurse.

\section{Path tracing}
We send a predetermined number of rays through random position in the pixels in raytraceImage. If the ray hits the diffuse surface, it sends out a random ray, biased towards the surface normal. We add all the contributions from surface shading and diffuse rays, averaged per pixel. We then apply tone mapping to the resulting color to get values between 0 and 1. The tone mapping we use is basically dividing the pixel values by the maximum intensity, and raising it to the power of some number less than one. There is an additional image of the cornell box rendered with 5k samples / pixel.

\section{SSE box intersection}
We implemented SSE box intersection. We intersect the root node with regular code. If there's an intersection, we call a second function intersectChildren(), which intersects the child objects (triangles) if it's a leaf node, or the child nodes if it's an internal node. For the intersections itself, we decided to implement a single ray-single box intersection where we intersect three planes in one operation. Then we do some inline assembly acrobatics to find the overlapping interval, by doing shifts and maxs.

\section{SSE triangle intersection}
In our BVH::build() function, in addition to building the tree, we store information about triangles in packets of four in a triangle cache in the leaf nodes. Basically, we find the vertex $A$, $B-A$, $C-A$, etc., and transpose them so each 128-bit vector has one component of each of these vertices, one component per triangle in the packet. So for instance, $A$ is stored as
$$
\left(
\begin{array}{cccc}
A_{x_1} A_{x_2} A_{x_3} A_{x_4}\\
A_{y_1} A_{y_2} A_{y_3} A_{y_4}\\
A_{z_1} A_{z_2} A_{z_3} A_{z_4}\\
\end{array}
\right)
$$
where the numerical subscript indicates which triangle it belongs to. We made our own cross product and dot product functions that calculates the dot and cross of two of these arrays. We calculate all the t's, betas and gamma values at once, then use SSE comparisons together with \_mm\_movemask\_ps to get the intersection results (intersection or no intersection) for each triangle. We then calculate P and N. Unfortunately, we had to calculate ALL the Ps and Ns because of the organization of our vectors, because transposing the vectors would be too expensive. 

Our tests indicate speedup in total rendering time using this SSE intersection code, but that includes all the serial parts of the program as well so the intersections alone might have a larger speedup. One thing to note is that we observed better results with more than 4 triangles in the leaf nodes, because we lower the chance of getting half-empty vectors. We found about 8 or 12 to be optimal.

\section{Non-axis aligned bounding boxes}
We researched non-axis aligned, or oriented, bounding boxes and found that the principal component analysis would be required to find the defining axes created by the triangle vertices. Building a covariance matrix of the points minus the center of the triangle yields a symmetric matrix that has real eigenvalues. The eigenvectors of this matrix are the axes of the oriented bounding box. We find the eigenvalues by taking the determinant of the matrix
$$
M=
\left(
\begin{array}{ccc}
a-\lambda & d & e\\ 
d & f-\lambda & f\\
e & f & c-\lambda\\
\end{array}
\right)
$$

This gives a cubic equation $\alpha \lambda^3 + \beta \lambda^2 + \gamma \lambda + \delta = 0$ where $\alpha =-3$, $\beta=3(a + b + c)$, $\gamma=3(e^2 + f^2 + d^2 - (a\cdot b + b\cdot c + a\cdot c)$ and $\delta=3(a\cdot b\cdot c - b\cdot e^2 -a\cdot f^2 - c\cdot d^2 + 2\cdot d\cdot e\cdot f)$.

We made a solver that finds the eigen vectors given the eigenvalues. This is implemented in Utility.cpp. It solves $Mv=\lambda v$ using gaussian elimination to reduce the matrix to reduced echelon form and outputs the normalized eigenvector. This can be found in Utility.cpp. Unfortunately, this is a far as we got since we did not have time to implement the cubic formula. The plan was to project the eigen vectors onto all the vertices to get the minimum and maximum points of the box. With the preceding information, we could change the optimized fast axis aligned slab to the regular plane intersection in the BVH tree traversal. 

\section{Statistics}
Here is our statistics for the scenes. All statistics are for 512x512 resolution unless otherwise stated. Also, the times are for our SSE implementation and are specified in seconds. For the ray/intersection statistics, we used 4 triangles per leaf node. The SSE implementation uses 8 triangles per leaf, as that seems to be more efficient.

\subsection{Rendering times}
\begin{tabular}{cccc}
{\bf Scene} & {\bf Shadows} & {\bf Rendering time} & {\bf Tree building time}\\
\hline
1 bunny     & Yes & 0.084176 & 0.848294\\
20 bunnies  & Yes & 0.115115 & 22.918213\\
Sponza      & Yes & 0.166750 & 0.747193\\
1 bunny     & No  & 0.054840 & 0.848294\\
20 bunnies  & No  & 0.065035 & 22.918213\\
Sponza      & No  & 0.162505 & 0.747193\\
\end{tabular}

\subsection{Ray, BVH and intersection statistics}
\begin{tabular}{cccccccc}
{\bf Scene} & {\bf Shadows}       & {\bf Ray-box}    & {\bf Ray-triangle}  & {\bf Total} & {\bf Ray-triangle}       & {\bf Total} & {\bf Leaf}\\
            &                     & {\bf intersects} & {\bf intersects}    & {\bf rays}  & {\bf intersects per ray} & {\bf nodes} & {\bf nodes}\\
\hline
1 bunny     & Yes                 & 1375045          & 566626              & 483840      & 1.17                     & 42881       & 21441\\
20 bunnies  & Yes                 & 12457832         & 3198190             & 495502      & 6.45                     & 876137      & 438069\\
Sponza      & Yes                 & 26828350         & 5415385             & 524288      & 10.33                    & 42645       & 21323\\
1 bunny     & No                  & 640985           & 256133              & 262144      & 0.98                     & 42881       & 21441\\
20 bunnies  & No                  & 5237842          & 979273              & 262144      & 3.74                     & 876137      & 438069\\
Sponza      & No                  & 14356648         & 2603678             & 262144      & 9.93                     & 42645       & 21323\\
Cornell\\
(16 samples)& Yes                 & 77354058         & 39393561            & 13319660    & 2.96                     & 21          & 11\\ 
Cornell\\
(100 samples)& Yes                & 386549458        & 214162592           & 81398851    & 2.63                     & 21          & 11\\ 
\end{tabular}

Note that the Cornell box images were rendered using multithreading, so the numbers are likely to be a bit off.

\end{document}
